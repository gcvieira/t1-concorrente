\documentclass[conference]{IEEEtran}
\IEEEoverridecommandlockouts
% The preceding line is only needed to identify funding in the first footnote. If that is unneeded, please comment it out.
\usepackage{cite}
\usepackage{amsmath,amssymb,amsfonts}
\usepackage{algorithmic}
\usepackage{graphicx}
\usepackage{textcomp}
\usepackage{xcolor}
\def\BibTeX{{\rm B\kern-.05em{\sc i\kern-.025em b}\kern-.08em
    T\kern-.1667em\lower.7ex\hbox{E}\kern-.125emX}}
\begin{document}

\title{Trabalho 1 - Filósofos Jantando\\Fundamentos de p. paralelo e distribuído}

\author{\IEEEauthorblockN{Guilherme Vieira, Lucca Demichei, Marcelo Rocha}
\IEEEauthorblockA{\textit{guilherme.camara@edu.pucrs.br, lucca.demichei@edu.pucrs.br, marcelo.rocha@edu.pucrs.br} \\
Escola Politécnica \\ PUCRS}}

\maketitle

\begin{abstract}
This document is a model and instructions for \LaTeX.
This and the IEEEtran.cls file define the components of your paper [title, text, heads, etc.]. *CRITICAL: Do Not Use Symbols, Special Characters, Footnotes, 
or Math in Paper Title or Abstract.
\end{abstract}

\begin{IEEEkeywords}
filósofos, concorrência, java, Dijkstra, thread, SMT.
\end{IEEEkeywords}

\section{O problema dos filósofos}

O problema dos filósofos jantando é um dos problemas clássicos usados para demonstrar os problemas existentes em um programa concorrente.
O problema consiste em cinco filósofos jantando em uma mesa circular, sendo um prato para cada filósofo e um garfo entre cada prato, onde os filósofos pensam e comem simultaneamente.
\cite{baeldung_java}

\section{Solução concorrente trivial}

aqui falamos sobre a solução inicial.

\section{Proteção contra deadlock}

aqui explicamos como o autor resolveu o deadlock.

\section{Análise de tempo e solução sem concorrência}

Analisamos o tempo de cada solução e comparamos com a nossa solução (sem concorrência). A nossa deve ser bem mais lenta .

\section{Screenshots}

Dump the screenshots here.

\section{Código fonte}

Make the source code available here.

\bibliographystyle{IEEEtran}
\bibliography{ref} % Entries are in the ref.bib file

\end{document}